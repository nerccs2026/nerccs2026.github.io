% ============================================================
% NERCCS 2026 Program Template (LaTeX)
% Single-file, robust, easy to edit last-minute.
% ============================================================

\documentclass[11pt]{article}
\usepackage{kpfonts}

% ---------- Page + typography ----------
\usepackage[margin=0.9in]{geometry}
\usepackage[T1]{fontenc}
%\usepackage{lmodern}
\usepackage{microtype}
\usepackage{graphicx}
\usepackage{xcolor}
\usepackage{lineno}
% ---------- Links (optional but useful) ----------
\usepackage[hidelinks]{hyperref}

% ---------- Lists / spacing ----------
\usepackage{enumitem}
\setlist[itemize]{leftmargin=*, itemsep=0.35em, topsep=0.35em}

% ---------- Tables (for overview grids, optional) ----------
\usepackage{tabularx}
\usepackage{longtable}
\usepackage{array}
\newcolumntype{Y}{>{\raggedright\arraybackslash}X}

% ---------- Headers / footers ----------
\usepackage{fancyhdr}
\pagestyle{fancy}
\fancyhf{}
\lhead{NERCCS 2026}
\rhead{Program}
\cfoot{\thepage}

% ---------- Section formatting ----------
\usepackage{titlesec}
\titleformat{\section}{\Large\bfseries}{}{0pt}{}
\titleformat{\subsection}{\large\bfseries}{}{0pt}{}
\titleformat{\subsubsection}{\normalsize\bfseries}{}{0pt}{}

% ---------- Convenience macros ----------
\newcommand{\NERCCS}{Northeast Regional Conference on Complex Systems (NERCCS 2026)}

% Conference header block
\newcommand{\ProgramHeader}[4]{%
	\begin{center}
		{\LARGE\bfseries #1\par}
		\vspace{0.35em}
		{\large #2\par}
		\vspace{0.15em}
		{\normalsize #3\par}
		\vspace{0.15em}
		{\normalsize\itshape #4\par}
	\end{center}
	\vspace{0.75em}
}

% A “time block” heading (e.g., Block 1)
\newcommand{\TimeBlock}[3]{%
	\section*{#1\hfill #2}
	\noindent\textit{#3}\par\vspace{0.4em}
	\hrule\vspace{0.8em}
}

% A parallel session (room track)
% Arguments: Room label, Session title, Chair (optional; leave blank if unknown)
\newcommand{\Session}[3]{%
	\subsection*{#1: #2}
	\if\relax\detokenize{#3}\relax\else
	\noindent Session Chair: #3\par
	\fi
	\vspace{0.3em}
}

% A talk line
% Arguments: Start time, Title, Author(s), Affiliation (optional), Notes (optional)
\newcommand{\Talk}[5]{%
	\item[\textbf{#1}] \textbf{#2}\\
	#3%
	\if\relax\detokenize{#4}\relax\else\ (\,#4\,)\fi%
	\if\relax\detokenize{#5}\relax\else\\{\small\itshape #5}\fi%
}

% For breaks / lunches
\newcommand{\BreakItem}[2]{%
	\noindent\textbf{#1}\hfill #2\par
}

% Optional: a compact overview grid for one page
% Use for attendee-scannable schedule summary.
\newcommand{\OverviewGrid}{%
	\section*{At-a-Glance Schedule (Overview)}
	\begin{tabularx}{\textwidth}{>{\bfseries}p{2.7cm} Y Y}
		\hline
		Time & Sloan Auditorium (Goergen 101) & Fantone Lecture Hall (Goergen 109) \\
		\hline
		09:00--10:00 & Block 1 (Session A) & Block 1 (Session B) \\
		10:15--11:15 & Block 2 (Session A) & Block 2 (Session B) \\
		11:30--12:30 & Block 3 (Session A) & Block 3 (Session B) \\
		14:00--15:00 & Block 4 (Session A) & Block 4 (Session B) \\
		15:15--16:15 & Block 5 (Session A) & Block 5 (Session B) \\
		16:30--17:30 & Block 6 (Session A) & Block 6 (Session B) \\
		\hline
	\end{tabularx}
	\vspace{1.0em}
}

% ---------- Document ----------
\begin{document}

%\linenumbers
	% EDIT THIS HEADER
	\ProgramHeader
	{\NERCCS}
	{Program Booklet}
	{Location: \textit{Goergen Hall, University of Rochester, 275 Hutchison Rd, 
	Rochester, NY 14620}\quad
		
	Dates: \textit{March 11--13, 2026}}
	
	
	\begin{center}
		
%	\textbf{\LARGE \textcolor{red}{DRAFT 2-21-26}}
	
	\end{center}
		
% TODO: \usepackage{graphicx} required
\begin{figure}
	\centering
	\includegraphics[width=0.7\linewidth]{nerccs2026logoProgram}

\end{figure}
	% Optional overview grid (comment out if not needed)
	% \OverviewGrid
	
	% ---------- Example: Day 1 Structure ----------
	\section*{Day 1 (March 11, 2026)}
	
	
	\BreakItem{Registration}{12:00 - 4:00}
	
	% ===== Block 1 =====
	\TimeBlock{Opening Remarks}{1:00 - 1:30}{}
	\TimeBlock{Keynote}{1:30-2:30}{Sloan Auditorium}
	\begin{itemize}[label={}, leftmargin=0pt]
		\Talk{}{The Physics and Chemistry of Synthesizing Abiotic Life in a Test Tube}{Juan Perez-Mercader}{Harvard University}{}
	\end{itemize}
	
	\BreakItem{Break}{2:30 - 2:45}
	


	
	
	\TimeBlock{Parallel Sessions 1}{2:45 - 3:45}{}
	\Session{Sloan Auditorium (Goergen 101)}{Network Control, Measurement, and Bias}{TBD}
	\begin{itemize}[label={}, leftmargin=0pt]
		\Talk{2:45}{Distributed Self-Control of Dynamical Networks by Adaptive Link 
		Weight Adjustments}{Hiroki Sayama}{Binghamton University, SUNY}{}
		\Talk{3:05}{Estimating Uncertainty in Network Measures and Structures 
		Arising from Noisy Data}{James Hartz}{University at Buffalo, SUNY}{}
		\Talk{3:25}{A Diagnostic Framework for Sampling-Induced Distortion in 
		Network Metrics}{Sriniwas Pandey}{Binghamton University, SUNY}{}		
	\end{itemize}
	
	\Session{Fantone Lecture Hall (Goergen 109)}{Adaptive Social Networks and Belief Dynamics}{TBD}
	\begin{itemize}[label={}, leftmargin=0pt]
		\Talk{2:45}{Bridging Simple and Complex Contagions in Belief Dynamics: A 
		Voter Model Extension}{Jo{\~a}o Franco}{Vermont Complex Systems Institute, 
		University of Vermont}{}
		\Talk{3:05}{Identifying brain network features that drive opinion formation 
		and changes in humans}{Shweta Hatote}{University at Buffalo, SUNY}{}
		\Talk{3:25}{Why We Experience Society Differently: Intrinsic Dispositions as 
		Drivers of Ideological Complexity in Adaptive Social Networks}{Akshay 
		Gangadhar}{Binghamton University}{}
		
	\end{itemize}
		
	\BreakItem{Break}{3:45 - 4:00}
	
	% ===== Block 2 =====
	\TimeBlock{Parallel Sessions 2}{4:00 - 5:00}{}

	
	\Session{Sloan Auditorium (Goergen 101)}{Information, Entropy, and Behavioral Signals}{}
	\begin{itemize}[label={}, leftmargin=0pt]
		\Talk{4:00}{On the Uniqueness of the Coupled Entropy and its 
		Applications}{Kenric P Nelson}{Photrek, Inc}{}
		\Talk{4:20}{A Network Perspective of the Stock Market Using Mutual 
		Information and Transfer Entropy}{Shangyi Bi}{Binghamton University, SUNY}{}
		\Talk{4:40}{Detecting Dynamic ``Fingerprints'' in Human Random 
		Keystrokes}{Jacqueline Blake}{Binghamton University, SUNY}{}
	\end{itemize}
	
	\Session{Fantone Lecture Hall (Goergen 109)}{Media, Platforms, and Field Mapping}{}
	\begin{itemize}[label={}, leftmargin=0pt]
		\Talk{4:00}{How News Connects:  Mapping the Semantic Topology of a Digital 
		Newspaper}{Sofia Sciangula}{Carlo Cattaneo University LIUC}{}
		\Talk{4:20}{Tempo-Dependent Emergence in Platform-Mediated Collective 
		Behavior}{Shaunette T. Ferguson}{Barnard College, Columbia University}{}
		\Talk{4:40}{Interactive Visualization of the Updated Complex Systems Keyword 
		Diagram}{Hiroki Sayama}{Binghamton University, SUNY}{}
	\end{itemize}
	
	\section*{\LARGE Day 2 (March 12)}
	
	% ===== Block 3 =====
	
	\TimeBlock{Keynote}{9:15-10:15}{}
	\begin{itemize}[label={}, leftmargin=0pt]
		\Talk{}{Title TBA}{Filippo Radicchi}{Indiana University}{Sloan Auditorium}
	\end{itemize}
	
		\BreakItem{Break}{10:15 - 10:30}
	\TimeBlock{Parallel Sessions 3}{10:30 - 11:30}{}
	
	\Session{Sloan Auditorium (Goergen 101)}{Social / Institutional Simulation and Innovation Dynamics}{TBD}
	\begin{itemize}[label={}, leftmargin=0pt]
		\Talk{10:30}{A Simulation Model for Historical Evolution of Geopolitical 
		Entities}{Andrea Agosta}{Carlo Cattaneo University LIUC}{}
		\Talk{10:50}{Using Agent Based Modeling to explore long-term economic 
		impacts 
		of nurse turnover in healthcare}{Kate J. O'Neill}{Binghamton 
		University, SUNY}{}
		\Talk{11:10}{Uncovering simultaneous breakthroughs with a robust measure of 
		disruptiveness}{Sadamori Kojaku}{Binghamton University, SUNY}{}
	\end{itemize}
	
	\Session{Fantone Lecture Hall (Goergen 109)}{Infrastructure, Risk, and Public/Health Networks}{TBD}
	\begin{itemize}[label={}, leftmargin=0pt]
		\Talk{10:30}{Graph-Based Analysis of Hurricane Stakeholder and Warning 
		Networks: Structural Roles, Information Flow, and Optimization 
		Diagnostics}{Hayford Adjavor}{Binghamton University, SUNY \& CenterPoint 
		Energy}{}
		\Talk{10:50}{Inaccessibility in Public Transit Networks}{Katherine Betz}{University at Buffalo, SUNY}{}
		\Talk{11:10}{Network-Driven Cohort Selection and Adverse Drug Reaction 
		Detection in Epilepsy Social Media}{Ziqi Guo}{Binghamton University, SUNY}{}
	\end{itemize}
	
	\TimeBlock{Lunch Break}{11:30 - 1:30}{Lunch on your own, see website for closeby options}
	
	\TimeBlock{Parallel Sessions 4}{1:30 - 2:30}{}
	
	\Session{Sloan Auditorium (Goergen 101)}{Graph Methods and Applied Dynamics}{}
	\begin{itemize}[label={}, leftmargin=0pt]
		\Talk{1:30}{distanceclosure: A Python Package for Network Sparsification 
		Based on Its Topology}{Robert Palermo}{Binghamton University, SUNY}{}
		\Talk{1:50}{The Effective Graph improves prediction of network dynamics in 
		biochemical models}{Xuan Wang}{Binghamton University, SUNY \& Indiana University}{}
		\Talk{2:10}{Robust Video Anomaly Detection under Partial Observation via 
		Patch-Motion Graph Smoothing}{Neda Amirirad}{Binghamton University, SUNY}{}
	\end{itemize}
	
	\Session{Fantone Lecture Hall (Goergen 109)}{Computational Models and Algorithm Diagnostics}{}
	\begin{itemize}[label={}, leftmargin=0pt]
		\Talk{1:30}{Diagnostic Evaluation of Conditional Generative Algorithms on 
		Curated Single Cell Reference Data}{Matthew Jehrio}{University of Rochester}{}
		\Talk{1:50}{Spark: Modular Spiking Neural Networks}{Mario Franco}{Binghamton University, SUNY}{}
		\Talk{2:10}{Scalability of Structure-Preserving Simulations in Plasma 
		Physics}{Sarthak Sharma}{University at Buffalo, SUNY}{}
	\end{itemize}
	
	\BreakItem{Break}{2:30 - 2:45}
	\TimeBlock{Keynote}{2:45 - 3:45}{Sloan Auditorium}
		\begin{itemize}[label={}, leftmargin=0pt]
		\Talk{}{Title TBA}{Adilson Motter}{Northwestern University}{}
	\end{itemize}
	\BreakItem{Break}{3:45 - 4:00}
	
	
	\TimeBlock{Parallel Sessions 5}{4:00 - 5:00}{}
	
	\Session{Sloan Auditorium (Goergen 101)}{Self-Organization, Robustness, and Epidemic Coupling}{TBD}
	\begin{itemize}[label={}, leftmargin=0pt]
		\Talk{4:00}{A Feedback-Driven Lyapunov Principle for Self-Organization in 
		Open Stochastic Systems}{Georgi Yordanov Georgiev}{Assumption University \& Worcester Polytechnic Institute}{}
		\Talk{4:20}{Canalization drives Robustness in the Evolution of Collective 
		Intelligence under Noise}{Srikanth Iyer}{Binghamton University, SUNY}{}
		\Talk{4:40}{A symbiotic SIR process}{Gerardo Palafox-Castillo}{Universidad Autonoma de Nuevo Leon}{}
	\end{itemize}
	
	\Session{Fantone Lecture Hall (Goergen 109)}{Collapse, Inequality, and Sustainability Dynamics}{TBD}
	\begin{itemize}[label={}, leftmargin=0pt]
		\Talk{4:00}{A Meta-Model of Endogenous Overcrowding-Driven Demographic 
		Collapse}{Federico Carucci}{Carlo Cattaneo University LIUC}{}
		\Talk{4:20}{Parasites become rich and drive inequality in approach to 
		collapse in a model of evolving networks}{Atiyab Zafar}{University of Delhi}{}
		\Talk{4:40}{Artificial Agents and Ecological Collapse: an LLM-enhanced 
		Agent-Based Model of a Sustainability Game}{Francesco Bertolotti}{Carlo Cattaneo University LIUC	}{}
	\end{itemize}
	
	\section*{\LARGE Day 3 (March 13)}
	
	\TimeBlock{Keynote}{9:15 - 10:15}{Sloan Auditorium}
	\begin{itemize}[label={}, leftmargin=0pt]
		\Talk{}{Title TBA}{Krishnan Padmanabhan }{University of Rochester}{}
	\end{itemize}
	
	\TimeBlock{Parallel Sessions 6}{10:30 - 11:30}{}
	
	\Session{Sloan Auditorium (Goergen 101)}{LLM Learning, Optimization, and Alignment}{TBD}
	\begin{itemize}[label={}, leftmargin=0pt]
		\Talk{10:30}{Prospect Learning in Large Language Models: Quantifying Risk 
		Preferences and Adaptation Dynamics}{Nency Dhameja}{Binghamton 
		University, SUNY}{}
		\Talk{10:50}{Evolutionary System-Prompt Optimization for LLM Agents in 
		Competitive Market Simulations}{Ossama Zaroual}{Carlo Cattaneo University LIUC}{}
		\Talk{11:10}{Cooperation or Defection? The Decay of Alignment Instructions in 
		Complex LLM Systems}{Andrea Monoli}{Carlo Cattaneo University LIUC}{}
	\end{itemize}
	
	\Session{Fantone Lecture Hall (Goergen 109)}{LLM Agents in Social Systems}{TBD}
	\begin{itemize}[label={}, leftmargin=0pt]
		\Talk{10:30}{Large Language Model Agents Partially Reproduce Real-World 
		Segregation Patterns in the Schelling Model}{Adreas Duus Pape}{Binghamton 
		University, SUNY}{}
		\Talk{10:50}{Going for the Third: Cooperation between LLM Agents in a 
		Three-Option Setting}{Leonardo Mascagni}{Carlo Cattaneo University LIUC}{}
		\Talk{11:10}{An Extension of the Sugarscape Model with LLM-based 
		Agents}{Luca Moroni}{Carlo Cattaneo University LIUC}{}
		
	\end{itemize}
	
	
		\TimeBlock{Closing Remarks}{11:50 - 12:00}{}

%	\OverviewGrid
	
	% ---------- End matter ----------
%	\vfill
%	\hrule
%	\vspace{0.5em}
%	{\small
%		\noindent\textbf{Notes for organizers:} This template is intentionally 
%		conservative (article class, simple macros) so it survives late edits.
%		Populate sessions with talk titles/authors; optionally add chairs and 
%		affiliations. If you want automated generation from CSV, keep the
%		\texttt{\textbackslash Talk} lines consistent and generate them 
%		programmatically.
%	}
	
\end{document}
